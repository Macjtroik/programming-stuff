\documentclass[11pt,letterpaper]{article}
\usepackage[utf8]{inputenc}
\usepackage{amsmath}
    \DeclareMathOperator{\arcsec}{arcsec}
    \DeclareMathOperator{\arccot}{arccot}
    \DeclareMathOperator{\arccsc}{arccsc}
\usepackage{amsfonts}
\usepackage{amssymb}
\usepackage{mathtools}
\usepackage{graphicx}
\usepackage[leftcaption]{sidecap}
\usepackage{setspace}
\usepackage{parskip}
\usepackage{gensymb}
\usepackage{mhchem}
\usepackage{multicol}
\usepackage{ragged2e}
\usepackage[left=2.54cm,right=2.54cm,top=2.54cm,bottom=2.54cm]{geometry}
\usepackage[normalem]{ulem}
\usepackage[scale]{tgheros}
\renewcommand\familydefault{\sfdefault}
\title{Propiedades de los gase}
\onehalfspacing
\begin{document}
\maketitle
\begin{flushleft}
\section{Unidades y proporcionalidad}
\subsection{Fuerz}
La constante de proporcionalidad para fuerza:\\
\[F= \frac{ma}{g_{c}}\hspace[1.5cm]\because F \propto ma\]
Deduciendo \(g_{c}\) para sistema internacional:\\
\begin{gather*}
    1N=\frac{1Kg \cdot 1\frac{m}{s^{2}}}{g_{c}}\\
    g_{c}=1 \frac{Kg \cdot m}{N \cdot s^{2}}
\end{gather*}
Para el sistema inglés:\\
\[ g_{c}=32.174\: \frac{lb_{m}\cdot ft}{lb_{f}\cdot s^{2}}\]

\subsection{Temperatura}
